\documentclass{article}

% Fonts
\usepackage{fontspec}

% General formatting
\usepackage[margin=1.3in, headheight=3ex, headsep=3ex]{geometry}
\usepackage{fancyhdr}
\usepackage{parskip}

% Colour
\usepackage[usenames,dvipsnames]{xcolor}

% Hyperlinks
\usepackage{url}
\usepackage{hyperref}

% Table stuff
\usepackage{booktabs}
\usepackage{longtable}


% Course details

\newcommand{\longcoursename}{
    Student Taught Courses (StuCo): Shilling the Rust Programming Language
}

\newcommand{\shortcoursename}{STUCO: RUSTLANG}
\newcommand{\courselocation}{WeH 5403}
\newcommand{\meetingstarttime}{19:00}
\newcommand{\meetingendtime}{19:50}
\newcommand{\meetingdays}{Wednesday}
\newcommand{\longsemester}{Spring 2023}
\newcommand{\shortsemester}{S23}
\newcommand{\academicyear}{2022-23}
\newcommand{\deptcode}{98}
\newcommand{\coursecode}{008}
\newcommand{\fullcoursecode}{\deptcode-\coursecode}

% Headers and Footers
\pagestyle{fancy}
\lhead{Pierce; Duvall}
\rhead{\fullcoursecode\ \shortsemester}

% Rust code
\usepackage{listings}
\definecolor{GrayCodeBlock}{RGB}{241,241,241}
\definecolor{BlackText}{RGB}{110,107,94}
\definecolor{OrangeTypename}{RGB}{182,86,17}
\definecolor{RedMacro}{RGB}{230,36,0}
\definecolor{GreenString}{RGB}{96,172,57}
\definecolor{PurpleKeyword}{RGB}{184,84,212}
\definecolor{GrayComment}{RGB}{170,170,170}
\definecolor{GoldDocumentation}{RGB}{180,165,45}
\definecolor{YellowAttrMacro}{RGB}{180,180,0}

%%
% Following component licensed as
% <https://github.com/denki/listings-rust/blob/master/LICENSE>:
%
% BSD 3-Clause License
% 
% Copyright (c) 2018, Tobias Denkinger
% All rights reserved.
% 
% Redistribution and use in source and binary forms, with or without
% modification, are permitted provided that the following conditions are met:
% 
% * Redistributions of source code must retain the above copyright notice, this
%   list of conditions and the following disclaimer.
% 
% * Redistributions in binary form must reproduce the above copyright notice,
%   this list of conditions and the following disclaimer in the documentation
%   and/or other materials provided with the distribution.
% 
% * Neither the name of the copyright holder nor the names of its
%   contributors may be used to endorse or promote products derived from
%   this software without specific prior written permission.
% 
% THIS SOFTWARE IS PROVIDED BY THE COPYRIGHT HOLDERS AND CONTRIBUTORS "AS IS"
% AND ANY EXPRESS OR IMPLIED WARRANTIES, INCLUDING, BUT NOT LIMITED TO, THE
% IMPLIED WARRANTIES OF MERCHANTABILITY AND FITNESS FOR A PARTICULAR PURPOSE ARE
% DISCLAIMED. IN NO EVENT SHALL THE COPYRIGHT HOLDER OR CONTRIBUTORS BE LIABLE
% FOR ANY DIRECT, INDIRECT, INCIDENTAL, SPECIAL, EXEMPLARY, OR CONSEQUENTIAL
% DAMAGES (INCLUDING, BUT NOT LIMITED TO, PROCUREMENT OF SUBSTITUTE GOODS OR
% SERVICES; LOSS OF USE, DATA, OR PROFITS; OR BUSINESS INTERRUPTION) HOWEVER
% CAUSED AND ON ANY THEORY OF LIABILITY, WHETHER IN CONTRACT, STRICT LIABILITY,
% OR TORT (INCLUDING NEGLIGENCE OR OTHERWISE) ARISING IN ANY WAY OUT OF THE USE
% OF THIS SOFTWARE, EVEN IF ADVISED OF THE POSSIBILITY OF SUCH DAMAGE.
%
\lstdefinelanguage{rust}{%
    sensitive,
    alsodigit={},
    alsoother={},
    alsoletter={!?},
    %
    % [1] reserved keywords
    % [2] traits
    % [3] primitive types
    % [4] type and value constructors
    % [5] identifier
    %
    % control flow
    morekeywords={
        break, continue, else, for, if, in, loop, match, return, while, await
    }, 
    % variable related
    morekeywords={as, const, let, move, mut, ref, static},
    % declaration related
    morekeywords={
        dyn, enum, fn, impl, Self, self, struct, trait, type, union, use, where
    },
    % modules
    morekeywords={crate, extern, mod, pub, super},
    % markers
    morekeywords={unsafe, async},
    % reserved identifiers
    morekeywords={
        abstract, alignof, become, box, do, final, macro, offsetof, override,
        priv, proc, pure, sizeof, typeof, unsized, virtual, yield
    },
    % grep 'pub trait [A-Za-z][A-Za-z0-9]*' -r . | sed 's/^.*pub trait \([A-Za-z][A-Za-z0-9]*\).*/\1/g' | sort -u | tr '\n' ',' | sed 's/^\(.*\),$/{\1}\n/g' | sed 's/,/, /g'
    morekeywords=[2]{
        Add, AddAssign, Any, AsciiExt, AsInner, AsInnerMut, AsMut, AsRawFd,
        AsRawHandle, AsRawSocket, AsRef, Binary, BitAnd, BitAndAssign, Bitor,
        BitOr, BitOrAssign, BitXor, BitXorAssign, Borrow, BorrowMut, Boxed,
        BoxPlace, BufRead, BuildHasher, CastInto, CharExt, Clone, CoerceUnsized,
        CommandExt, Copy, Debug, DecodableFloat, Default, Deref, DerefMut,
        DirBuilderExt, DirEntryExt, Display, Div, DivAssign,
        DoubleEndedIterator, DoubleEndedSearcher, Drop, EnvKey, Eq, Error,
        ExactSizeIterator, ExitStatusExt, Extend, FileExt, FileTypeExt, Float,
        Fn, FnBox, FnMut, FnOnce, Freeze, From, FromInner, FromIterator,
        FromRawFd, FromRawHandle, FromRawSocket, FromStr, FullOps,
        FusedIterator, Generator, Hash, Hasher, Index, IndexMut, InPlace, Int,
        Into, IntoCow, IntoInner, IntoIterator, IntoRawFd, IntoRawHandle,
        IntoRawSocket, IsMinusOne, IsZero, Iterator, JoinHandleExt, LargeInt,
        LowerExp, LowerHex, MetadataExt, Mul, MulAssign, Neg, Not, Octal,
        OpenOptionsExt, Ord, OsStrExt, OsStringExt, Packet, PartialEq,
        PartialOrd, Pattern, PermissionsExt, Place, Placer, Pointer, Product,
        Put, RangeArgument, RawFloat, Read, Rem, RemAssign, Seek, Shl,
        ShlAssign, Shr, ShrAssign, Sized, SliceConcatExt, SliceExt, SliceIndex,
        Stats, Step, StrExt, Sub, SubAssign, Sum, Sync, TDynBenchFn, Terminal,
        Termination, ToOwned, ToSocketAddrs, ToString, Try, TryFrom, TryInto,
        UnicodeStr, Unsize, UpperExp, UpperHex, WideInt, Write
    },
    % additional traits
    morekeywords=[2]{Send, Unpin},
    % primitive types
    morekeywords=[3]{
        bool, char, f32, f64, i8, i16, i32, i64, isize, str, u8, u16, u32, u64,
        unit, usize, i128, u128
    },
    % prelude value constructors
    morekeywords=[4]{Err, false, None, Ok, Some, true},  
    % grep 'pub \(type\|struct\|enum\) [A-Za-z][A-Za-z0-9]*' -r . | sed 's/^.*pub \(type\|struct\|enum\) \([A-Za-z][A-Za-z0-9]*\).*/\2/g' | sort -u | tr '\n' ',' | sed 's/^\(.*\),$/{\1}\n/g' | sed 's/,/, /g'    
    morekeywords=[3]{
        AccessError, Adddf3, AddI128, AddoI128, AddoU128, ADDRESS, ADDRESS64,
        addrinfo, ADDRINFOA, AddrParseError, Addsf3, AddU128, advice, aiocb,
        Alignment, AllocErr, AnonPipe, Answer, Arc, Args, ArgsInnerDebug,
        ArgsOs, Argument, Arguments, ArgumentV1, Ashldi3, Ashlti3, Ashrdi3,
        Ashrti3, AssertParamIsClone, AssertParamIsCopy, AssertParamIsEq,
        AssertUnwindSafe, AtomicBool, AtomicPtr, Attr, auxtype, auxv, BackPlace,
        BacktraceContext, Barrier, BarrierWaitResult, Bencher, BenchMode,
        BenchSamples, BinaryHeap, BinaryHeapPlace, blkcnt, blkcnt64, blksize,
        BOOL, boolean, BOOLEAN, BoolTrie, BorrowError, BorrowMutError, Bound,
        Box, bpf, BTreeMap, BTreeSet, Bucket, BucketState, Buf, BufReader,
        BufWriter, Builder, BuildHasherDefault, BY, BYTE, Bytes,
        CannotReallocInPlace, cc, Cell, Chain, CHAR, CharIndices,
        CharPredicateSearcher, Chars, CharSearcher, CharsError,
        CharSliceSearcher, CharTryFromError, Child, ChildPipes, ChildStderr,
        ChildStdin, ChildStdio, ChildStdout, Chunks, ChunksMut, ciovec, clock,
        clockid, Cloned, cmsgcred, cmsghdr, CodePoint, Color, ColorConfig,
        Command, CommandEnv, Component, Components, CONDITION, condvar, Condvar,
        CONSOLE, CONTEXT, Count, Cow, cpu, CRITICAL, CStr, CString,
        CStringArray, Cursor, Cycle, CycleIter, daddr, DebugList, DebugMap,
        DebugSet, DebugStruct, DebugTuple, Decimal, Decoded, DecodeUtf16,
        DecodeUtf16Error, DecodeUtf8, DefaultEnvKey, DefaultHasher, dev, device,
        Difference, Digit32, DIR, DirBuilder, dircookie, dirent, dirent64,
        DirEntry, Discriminant, DISPATCHER, Divdf3, Divdi3, Divmoddi4,
        Divmodsi4, Divsf3, Divsi3, Divti3, dl, Dl, Dlmalloc, Dns, DnsAnswer,
        DnsQuery, dqblk, Drain, DrainFilter, Dtor, Duration, DwarfReader, DWORD,
        DWORDLONG, DynamicLibrary, Edge, EHAction, EHContext, Elf32, Elf64,
        Empty, EmptyBucket, EncodeUtf16, EncodeWide, Entry, EntryPlace,
        Enumerate, Env, epoll, errno, Error, ErrorKind, EscapeDebug,
        EscapeDefault, EscapeUnicode, event, Event, eventrwflags, eventtype,
        ExactChunks, ExactChunksMut, EXCEPTION, Excess, ExchangeHeapSingleton,
        exit, exitcode, ExitStatus, Failure, fd, fdflags, fdsflags, fdstat, ff,
        fflags, File, FILE, FileAttr, filedelta, FileDesc, FilePermissions,
        filesize, filestat, FILETIME, filetype, FileType, Filter, FilterMap,
        Fixdfdi, Fixdfsi, Fixdfti, Fixsfdi, Fixsfsi, Fixsfti, Fixunsdfdi,
        Fixunsdfsi, Fixunsdfti, Fixunssfdi, Fixunssfsi, Fixunssfti, Flag,
        FlatMap, Floatdidf, FLOATING, Floatsidf, Floatsisf, Floattidf,
        Floattisf, Floatundidf, Floatunsidf, Floatunsisf, Floatuntidf,
        Floatuntisf, flock, ForceResult, FormatSpec, Formatted, Formatter, Fp,
        FpCategory, fpos, fpos64, fpreg, fpregset, FPUControlWord, Frame,
        FromBytesWithNulError, FromUtf16Error, FromUtf8Error, FrontPlace,
        fsblkcnt, fsfilcnt, fsflags, fsid, fstore, fsword, FullBucket,
        FullBucketMut, FullDecoded, Fuse, GapThenFull, GeneratorState, gid,
        glob, glob64, GlobalDlmalloc, greg, group, GROUP, Guard, GUID, Handle,
        HANDLE, Handler, HashMap, HashSet, Heap, HINSTANCE, HMODULE, hostent,
        HRESULT, id, idtype, if, ifaddrs, IMAGEHLP, Immut, in, in6, Incoming,
        Infallible, Initializer, ino, ino64, inode, input, InsertResult,
        Inspect, Instant, int16, int32, int64, int8, integer, IntermediateBox,
        Internal, Intersection, intmax, IntoInnerError, IntoIter,
        IntoStringError, intptr, InvalidSequence, iovec, ip, IpAddr, ipc,
        Ipv4Addr, ipv6, Ipv6Addr, Ipv6MulticastScope, Iter, IterMut, itimerspec,
        itimerval, jail, JoinHandle, JoinPathsError, KDHELP64, kevent, kevent64,
        key, Key, Keys, KV, l4, LARGE, lastlog, launchpad, Layout, Lazy, lconv,
        Leaf, LeafOrInternal, Lines, LinesAny, LineWriter, linger, linkcount,
        LinkedList, load, locale, LocalKey, LocalKeyState, Location, lock,
        LockResult, loff, LONG, lookup, lookupflags, LookupHost, LPBOOL, LPBY,
        LPBYTE, LPCSTR, LPCVOID, LPCWSTR, LPDWORD, LPFILETIME, LPHANDLE,
        LPOVERLAPPED, LPPROCESS, LPPROGRESS, LPSECURITY, LPSTARTUPINFO, LPSTR,
        LPVOID, LPWCH, LPWIN32, LPWSADATA, LPWSAPROTOCOL, LPWSTR, Lshrdi3,
        Lshrti3, lwpid, M128A, mach, major, Map, mcontext, Metadata, Metric,
        MetricMap, mflags, minor, mmsghdr, Moddi3, mode, Modsi3, Modti3,
        MonitorMsg, MOUNT, mprot, mq, mqd, msflags, msghdr, msginfo, msglen,
        msgqnum, msqid, Muldf3, Mulodi4, Mulosi4, Muloti4, Mulsf3, Multi3, Mut,
        Mutex, MutexGuard, MyCollection, n16, NamePadding, NativeLibBoilerplate,
        nfds, nl, nlink, NodeRef, NoneError, NonNull, NonZero, nthreads,
        NulError, OccupiedEntry, off, off64, oflags, Once, OnceState,
        OpenOptions, Option, Options, OptRes, Ordering, OsStr, OsString, Output,
        OVERLAPPED, Owned, Packet, PanicInfo, Param, ParseBoolError,
        ParseCharError, ParseError, ParseFloatError, ParseIntError, ParseResult,
        Part, passwd, Path, PathBuf, PCONDITION, PCONSOLE, Peekable, PeekMut,
        Permissions, PhantomData, pid, Pin, Pipes, PlaceBack, PlaceFront, PLARGE,
        PoisonError, pollfd, PopResult, port, Position, Powidf2, Powisf2,
        Prefix, PrefixComponent, PrintFormat, proc, Process, PROCESS,
        processentry, protoent, PSRWLOCK, pthread, ptr, ptrdiff, PVECTORED,
        Queue, radvisory, RandomState, Range, RangeFrom, RangeFull,
        RangeInclusive, RangeMut, RangeTo, RangeToInclusive, RawBucket, RawFd,
        RawHandle, RawPthread, RawSocket, RawTable, RawVec, Rc, ReadDir,
        Receiver, recv, RecvError, RecvTimeoutError, ReentrantMutex,
        ReentrantMutexGuard, Ref, RefCell, RefMut, REPARSE, Repeat, Result, Rev,
        Reverse, riflags, rights, rlim, rlim64, rlimit, rlimit64, roflags, Root,
        RSplit, RSplitMut, RSplitN, RSplitNMut, RUNTIME, rusage, RwLock, RWLock,
        RwLockReadGuard, RwLockWriteGuard, sa, SafeHash, Scan, sched, scope,
        sdflags, SearchResult, SearchStep, SECURITY, SeekFrom, segment, Select,
        SelectionResult, sem, sembuf, send, Sender, SendError, servent, sf,
        Shared, shmatt, shmid, ShortReader, ShouldPanic, Shutdown, siflags,
        sigaction, SigAction, sigevent, sighandler, siginfo, Sign, signal,
        signalfd, SignalToken, sigset, sigval, Sink, SipHasher, SipHasher13,
        SipHasher24, size, SIZE, Skip, SkipWhile, Slice, SmallBoolTrie,
        sockaddr, SOCKADDR, sockcred, Socket, SOCKET, SocketAddr, SocketAddrV4,
        SocketAddrV6, socklen, speed, Splice, Split, SplitMut, SplitN,
        SplitNMut, SplitPaths, SplitWhitespace, spwd, SRWLOCK, ssize, stack,
        STACKFRAME64, StartResult, STARTUPINFO, stat, Stat, stat64, statfs,
        statfs64, StaticKey, statvfs, StatVfs, statvfs64, Stderr, StderrLock,
        StderrTerminal, Stdin, StdinLock, Stdio, StdioPipes, Stdout, StdoutLock,
        StdoutTerminal, StepBy, String, StripPrefixError, StrSearcher,
        subclockflags, Subdf3, SubI128, SuboI128, SuboU128, subrwflags,
        subscription, Subsf3, SubU128, Summary, suseconds, SYMBOL, SYMBOLIC,
        SymmetricDifference, SyncSender, sysinfo, System, SystemTime,
        SystemTimeError, Take, TakeWhile, tcb, tcflag, TcpListener, TcpStream,
        TempDir, TermInfo, TerminfoTerminal, termios, termios2, TestDesc,
        TestDescAndFn, TestEvent, TestFn, TestName, TestOpts, TestResult,
        Thread, threadattr, threadentry, ThreadId, tid, time, time64, timespec,
        TimeSpec, timestamp, timeval, timeval32, timezone, tm, tms, ToLowercase,
        ToUppercase, TraitObject, TryFromIntError, TryFromSliceError, TryIter,
        TryLockError, TryLockResult, TryRecvError, TrySendError, TypeId, U64x2,
        ucontext, ucred, Udivdi3, Udivmoddi4, Udivmodsi4, Udivmodti4, Udivsi3,
        Udivti3, UdpSocket, uid, UINT, uint16, uint32, uint64, uint8, uintmax,
        uintptr, ulflags, ULONG, ULONGLONG, Umoddi3, Umodsi3, Umodti3,
        UnicodeVersion, Union, Unique, UnixDatagram, UnixListener, UnixStream,
        Unpacked, UnsafeCell, UNWIND, UpgradeResult, useconds, user, userdata,
        USHORT, Utf16Encoder, Utf8Error, Utf8Lossy, Utf8LossyChunk,
        Utf8LossyChunksIter, utimbuf, utmp, utmpx, utsname, uuid, VacantEntry,
        Values, ValuesMut, VarError, Variables, Vars, VarsOs, Vec, VecDeque, vm,
        Void, WaitTimeoutResult, WaitToken, wchar, WCHAR, Weak, whence, WIN32,
        WinConsole, Windows, WindowsEnvKey, winsize, WORD, Wrapping, wrlen,
        WSADATA, WSAPROTOCOL, WSAPROTOCOLCHAIN, Wtf8, Wtf8Buf, Wtf8CodePoints,
        xsw, xucred, Zip, zx
    },
    % prelude macros
    morekeywords=[5]{
        assert!, assert_eq!, assert_ne!, cfg!, column!, compile_error!, concat!,
        concat_idents!, debug_assert!, debug_assert_eq!, debug_assert_ne!, env!,
        eprint!, eprintln!, file!, format!, format_args!, include!,
        include_bytes!, include_str!, line!, module_path!, option_env!, panic!,
        print!, println!, select!, stringify!, thread_local!, try!,
        unimplemented!, unreachable!, vec!, write!, writeln!, ?,
    },
% end licensed code
    columns=fullflexible,
    keepspaces=true,
    frame=single,
    framesep=0pt,
    framerule=0pt,
    framexleftmargin=4pt,
    framexrightmargin=4pt,
    framextopmargin=5pt,
    framexbottommargin=3pt,
    xleftmargin=4pt,
    xrightmargin=4pt,
    backgroundcolor=\color{GrayCodeBlock},
    basicstyle=\ttfamily\color{BlackText},
    % [1] reserved keywords
    % [2] traits
    % [3] primitive types
    % [4] type and value constructors
    % [5] identifier
    keywordstyle=\bfseries\color{PurpleKeyword},
    keywordstyle=[2]\color{BlueViolet},
    keywordstyle=[3]\color{BlueViolet},
    keywordstyle=[4]\color{OrangeTypename},
    keywordstyle=[5]\color{RedMacro},
    comment=[l][\color{GrayComment}\slshape]{//},
    morecomment=[s][\color{GrayComment}\slshape]{/*}{*/},
    morecomment=[l][\color{GoldDocumentation}\slshape]{///},
    morecomment=[s][\color{GoldDocumentation}\slshape]{/*!}{*/},
    morecomment=[l][\color{GoldDocumentation}\slshape]{//!},
    moredelim=[s][{\itshape\bfseries\color{YellowAttrMacro}}]{\#[}{]},
    moredelim=[s][{\itshape\bfseries\color{YellowAttrMacro}}]{\#![}{]},
    stringstyle=\color{GreenString},
    string=[b]"
}%


\lstnewenvironment{codeblock}[1][]{\lstset{language=rust,#1}}{}

% code inline
\newcommand{\code}[2][]{{\sloppy
\ifmmode
\text{\lstinline[language=rust,#1]`#2`}
\else
{\lstinline[language=rust,#1]`#2`}%
\fi}}


\begin{document}

\thispagestyle{empty}
\begin{center}
\begin{minipage}{.85\textwidth}
    \centering
    {\huge {\fullcoursecode: \longcoursename}}
    {\small \textit{Updated \today}}

    \vspace{1em}

    \begin{tabular}{@{}rl@{}}
        Cooper Pierce & \href{mailto:cppierce@andrew.cmu.edu}{\texttt{cppierce@andrew.cmu.edu}} \\ 
        Jack Duvall & \href{mailto:jrduvall@andrew.cmu.edu}{\texttt{jrduvall@andrew.cmu.edu}} \\
    \end{tabular}

    \vspace{1em}

    \begin{center}
        For general questions or concerns any instructor can address, please use 
        \href{mailto:rust-stuco-staff@lists.andrew.cmu.edu}{\texttt{rust-stuco-staff@lists.andrew.cmu.edu}}
    \end{center}

    \vspace{1em}

    \longsemester
\end{minipage}
\end{center}

\vspace{3em}


\begin{tabular*}{.93\textwidth}{@{\extracolsep{\fill}}ll}
    \toprule
    Office Hours: by appointment & Web: \url{https://rust-stuco.github.io} \\
    Time: \meetingdays\ \meetingstarttime-\meetingendtime & Location: \courselocation \\
    \bottomrule
\end{tabular*}

\vspace{5em}

\section*{Course Description}

This course is an introduction to the Rust programming language, a memory-safe
systems programming language. Topics covered include memory safety, ownership \&
lifetimes, the Rust stdlib, concurrency, traits, macros, and unsafe. Outside of
class, there will be basic programming exercises and during the second half of
the semester, a student-defined project.

This course assumes knowledge of topics covered by 15--122/15--150. Knowledge of
the systems programming related topics covered in 15--213 is a bonus, but not
required.

This course is taught in Rust, and you are not expected to have any prior
exposure to Rust. However, we may refer to examples written in C and/or SML
throughout the class to compare different approaches.


\section*{Learning Objectives}

Successful students in this course will

\begin{itemize}
    \item be able to read and write Rust code which includes:
        \begin{itemize}
            \item core language constructs such as: structs, enums, and impl
                blocks; ownership, borrowing, and lifetimes; pattern matching;
                traits; polymorphism, including trait objects
            \item common standard library types and functions, e.g.,
                \code{Option<T>}, \code{Result<T, E>}, \code{Vec<T>},
                \code{String} and \code{&str}
            \item common standard library traits, e.g., \code{Iterator<Item =
                T>}, \code{Copy}, \code{Clone}, \code{Debug}
        \end{itemize}
    \item understand some intermediate to advanced Rust topics such as:
            \begin{itemize}
                \item declarative macros
                \item Rust's approaches to concurrency, parallelis and memory
                    consistency
                \item \code{unsafe} semantics
            \end{itemize}
    \item complete a substantial (6-9 hour) project in Rust either individually,
        or with a partner
    \item be able to use tools common in the Rust ecosystem, including
        Cargo, \texttt{rustfmt}, \texttt{clippy}, and \texttt{rustdoc}
    \item be able to locate and read Rust documentation
\end{itemize}

\section*{Additional Materials}

Access to a computer able to run \texttt{rustc} (the Rust compiler) and Cargo
(the standard build system) is required for this course. We'll go over how to
set these up in class.

\section*{Attendance Policy}

Attendance is mandatory. \textbf{As per StuCo guidelines, if you have three
unexcused absences, you will automatically fail this course}.

\section*{Grading}

The final course grade will be calculated using the following categories. The
course is out of 1000 points.

\begin{center}
    \begin{tabular}{@{}lr@{}}
        Assessment Type     & Total Points \\ \toprule
        Participation       &          500 \\
        Homework            &          100 \\
        Midterm             &           50 \\
        Final               &          350 \\ \midrule
        Total               &         1000 \\ \bottomrule
    \end{tabular}
\end{center}

Grades will be assigned as follows, based on the number of points earned.

\begin{center}
    \begin{tabular}{@{}ll@{}}
        Grade    & Points    \\ \toprule
        P (or S) & 600--1000 \\
        NP       & 0--600    \\ \bottomrule
    \end{tabular}
\end{center}

\subsection*{Participation}

Attendance is mandatory. As per StuCo guidelines, if you have three unexcused
absences, you will automatically fail this course. We will take attendance with
a sign-in sheet or an interactive activity.

If you need to miss a session for a reason we can reasonably excuse contact us
and we will determine an appropriate alternative so you don't miss out on that
week's content.

\subsection*{Homework}
 
Homeworks will consist primarily of writing Rust code applying topics we've
discussed in lecture.

Homeworks are designed to not take more than an hour (if they are, talk to us!),
and be done between the meeting after which they're released and the next
meeting.

We expect you to turn in a submission for every homework, but they are graded
for effort, not correctness. This means that you should attempt to correctly
complete every homework, but if you put in a good faith effort and do not manage
to finish it correctly, you will still receive credit.

\subsection*{Midterm}

You will be choosing your project for the final. Individually or in a team of
two people, you will be expected to include a short writeup of what you plan to
do, including details about how you will structure your code and what
third-party libraries you plan on using. Subsequent changes to this are okay,
but we want you to have a good foundation from which to complete a project.

\subsection*{Final}

You will be completing a medium-sized programming project in Rust using what
you've learned from the course. The project will showcase Rust's elegance for
systems engineering and teach you valuable life lessons.

\subsection*{Late work}

All homework released prior to the midterm will be due when the midterm is, and
similarly for the final. Beyond this, no late work is accepted.

If you experience extenuating circumstances (e.g., you are hospitalized, have a
family emergency, a conflict with a university-sponsored event) that prohibit
you from submitting your assignments on time, please let us know (in advance, if
possible). We'll evaluate these instances on a case-by-case basis and arrange
for reasonable accommodations.

\section*{Academic Integrity}

Discussion of homework problems, written or verbal, is always permitted. That
said, verbatim or like-verbatim copying of answers over any medium is expressly
forbidden.

For the final project, external dependencies are allowed (so long as you
acknowledge them, though an entry in a manifest file like \texttt{Cargo.toml}
will suffice), and we'll discuss what's appropriate there during project
proposals.

\section*{Course Schedule}

\begin{longtable}{rp{.5\textwidth}@{\hskip .05\textwidth}p{.3\textwidth}}
    Date        & Topic(s)                          & Assignments Released  \\
    \toprule
    2023-01-18  & Tour of Rust; Motivation;
                  \code{struct}s, \code{enum}s, basic syntax
                & Rust installation and setup                               \\
    2023-01-25  & Pattern matching, \code{impl} blocks; introduction to
                  ownership, lifetimes and the borrow system
                & PPM Parsing                                               \\
    2023-02-01  & Polymorphism, bounded (traits) and unbounded; trait objects
                  and dynamic dispatch
                & Parsing Combinator Library                                \\
    2023-02-08  & Function types, closures; more advanced ownership semantics
                &                                                           \\
    2023-02-15  & Livecoding a library for text iteration: applying lifetimes,
                  polymorphism, traits.
                & Midterm                                                   \\
    2023-02-22  & The standard library and string handling
                & \texttt{egrep} clone                                      \\
    2023-03-01  & Error handling: \code{panic!}, \code{?}, testing: unit tests,
                  property based testing
                  (\href{https://github.com/AltSysrq/proptest}{\textit{proptest}})
                  and fuzzing
                  (\href{https://github.com/rust-fuzz/cargo-fuzz}{\textit{cargo-fuzz}})
                & Note: all prior released assignments due at this point    \\
    2023-03-08  & Spring Break (no class)
                & ---                                                       \\
    2023-03-15  & Macros: built-in, declarative, procedural
                & \code{#[derive(Map)]} \textit{(optional)}                 \\
    2023-03-22  & \code{unsafe}: \begin{itemize}
                        \item Why have \code{unsafe}?
                        \item Raw pointers
                        \item \code{UnsafeCell}
                        \item \code{unsafe} functions and traits
                        \item \code{static mut}
                        \item FFI (\code{extern},
                            \href{https://github.com/rust-lang/rust-bindgen}{\textit{bindgen}},
                            \href{https://github.com/dtolnay/cxx}{\textit{cxx}})
                    \end{itemize}
                & ---                                                       \\
    2023-03-29  & Concurrency/Parallelism 1: threads, mutexes,
                  \code{Sync}, \code{Send}
                & ---                                                       \\
    2023-04-05  & Concurrency/Parallelism 2: \code{Pin}, \code{async},
                  \code{await}, futures, streams, polling,
                  \href{https://tokio.rs/}{\textit{tokio}}
                & ---                                                       \\
    2023-04-12  & Concurrency/Parallelism 3: atomic variables, memory orderings
                  and consistency, cmpxchg/cas and ll/sc
                & ---                                                       \\
    2023-04-19  & Advanced Topics or Guest Lecture (TBA)
                & ---                                                       \\
    2023-04-26  & Project presentations
                & Note: all assignments due by this point                   \\
    \bottomrule
\end{longtable}

\section*{Course Note}

\subsection*{Academic Integrity \& Collaboration}

\textit{From CMU's Policy on Academic Integrity:} In any manner of presentation,
it is the responsibility of each student to produce her/his own original
academic work. Collaboration or assistance on academic work to be graded is not
permitted unless explicitly authorized by the course instructor(s). Students may
utilize the assistance provided by Academic Development, the Global
Communication Center, and the Academic Resource Center (CMU-Q) unless
specifically prohibited by the course instructor(s). Any other sources of
collaboration or assistance must be specifically authorized by the course
instructor(s).
 
In all academic work to be graded, the citation of all sources is required. When
collaboration or assistance is permitted by the course instructor(s) or when a
student utilizes the services provided by Academic Development, the Global
Communication Center, and the Academic Resource Center (CMU-Q), the
acknowledgement of any collaboration or assistance is likewise required. This
citation and acknowledgement must be incorporated into the work submitted and
not separately or at a later point in time. Failure to do so is dishonest and is
subject to disciplinary action.

Instructors have a duty to communicate their expectations including those
specific to collaboration, assistance, citation and acknowledgement within each
course. Students likewise have a duty to ensure that they understand and abide
by the standards that apply in any course or academic activity. In the absence
of such understanding, it is the student’s responsibility to seek additional
information and clarification.

\subsection*{Accommodations for students with disabilities}

If you have a disability and have an accommodations letter from the Disability
Resources office, I encourage you to discuss your accommodations and needs with
me as early in the semester as possible. I will work with you to ensure that
accommodations are provided as appropriate. If you suspect that you may have a
disability and would benefit from accommodations but are not yet registered with
the Office of Disability Resources, I encourage you to contact them at
\href{mailto:access@andrew.cmu.edu}{\texttt{access@andrew.cmu.edu}}.


\subsection*{Statement on student wellness}

Take care of yourself.  Do your best to maintain a healthy lifestyle this
semester by eating well, exercising, avoiding drugs and alcohol, getting enough
sleep and taking some time to relax. This will help you achieve your goals and
cope with stress.

All of us benefit from support during times of struggle. There are many helpful
resources available on campus and an important part of the college experience is
learning how to ask for help. Asking for support sooner rather than later is
almost always helpful.

If you or anyone you know experiences any academic stress, difficult life
events, or feelings like anxiety or depression, we strongly encourage you to
seek support. Counseling and Psychological Services (CaPS) is here to help: call
412-268-2922 and visit their website at \url{http://www.cmu.edu/counseling/}.
Consider reaching out to a friend, faculty or family member you trust for help
getting connected to the support that can help.

\subsection*{Statement on Diversity, Equity and Inclusion}

We must treat every individual with respect. We are diverse in many ways, and
this diversity is fundamental to building and maintaining an equitable and
inclusive campus community. Diversity can refer to multiple ways that we
identify ourselves, including but not limited to race, color, national origin,
language, sex, disability, age, sexual orientation, gender identity, religion,
creed, ancestry, belief, veteran status, or genetic information. Each of these
diverse identities, along with many others not mentioned here, shape the
perspectives our students, faculty, and staff bring to our campus. We, at CMU,
will work to promote diversity, equity and inclusion not only because diversity
fuels excellence and innovation, but because we want to pursue justice. We
acknowledge our imperfections while we also fully commit to the work, inside and
outside of our classrooms, of building and sustaining a campus community that
increasingly embraces these core values.

Each of us is responsible for creating a safer, more inclusive environment.

Unfortunately, incidents of bias or discrimination do occur, whether intentional
or unintentional. They contribute to creating an unwelcoming environment for
individuals and groups at the university. Therefore, the university encourages
anyone who experiences or observes unfair or hostile treatment on the basis of
identity to speak out for justice and support, within the moment of the incident
or after the incident has passed. Anyone can share these experiences using the
following resources:

\begin{itemize}
    \item Center for Student Diversity and Inclusion: 
        \href{mailto:csdi@andrew.cmu.edu}{csdi@andrew.cmu.edu},
        (412) 268-2150
    \item Report-It online anonymous reporting platform: \url{reportit.net}
    \begin{itemize}
        \item username: tartans
        \item password: plaid
    \end{itemize}
\end{itemize}

All reports will be documented and deliberated to determine if there should be
any following actions. Regardless of incident type, the university will use all
shared experiences to transform our campus climate to be more equitable and
just.

\end{document}
